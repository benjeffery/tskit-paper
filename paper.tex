\documentclass{article}
\usepackage[round]{natbib}
\usepackage{listings}
\usepackage[english]{babel}%
\usepackage[T1]{fontenc}%
\usepackage[utf8]{inputenc}%
\usepackage{amsmath,amssymb,amsfonts}%
\usepackage{geometry}%
\usepackage{color}
\usepackage{graphicx}
\usepackage{dsfont}
\usepackage{verbatim}%
\usepackage{environ}%
\usepackage[right]{lineno}%
\usepackage{nameref}
%\usepackage{showkeys}

% local definitions
\newcommand{\msprime}[0]{\texttt{msprime}}
\newcommand{\tskit}[0]{\texttt{tskit}}
\newcommand{\SLiM}[0]{\texttt{SLiM}}
\newcommand{\fwdpy}[0]{\texttt{fwdpy11}}

\newcommand{\jkcomment}[1]{\textcolor{red}{#1}}

\begin{document}

\title{Tskit: a portable library for population scale genealogical analysis}
\author{Author list to be filled in
}
% \address{

%%Affiliations:
% Franz Baumdicker:
% Cluster of Excellence "Controlling Microbes to Fight Infections", Mathematical and Computational Population Genetics, University of Tübingen, Germany


% \section*{Contact:} \href{jerome.kelleher@bdi.ox.ac.uk}{jerome.kelleher@bdi.ox.ac.uk}

\maketitle

\begin{abstract}
The ability to store and analyse related genetic sequences is an
essential requirement for simulation, inference and analysis in both
population genetics and phlyogenetics. The recent introduction of the
succinct tree sequence data structure has provided a way to achieve
this at population scale. Here we present the \tskit\ software,
a high-performance, portable, open-source, community-driven library.
\tskit\ allows the creation, manipulation and analysis of succinct tree
sequences, with first-class support for provenance and user-defined metadata.
\tskit\ enables a common foundation across software projects that use
succinct tree sequences, which results in unprecedented interoperability,
reproducibility and maintainability.
\end{abstract}


\textbf{Keywords:} Tree sequences, Python

\section*{Introduction}

\citep{kelleher2018efficient}

\section*{Results}


\section*{Discussion}

\section*{Acknowledgments}

\bibliographystyle{plainnat}
\bibliography{paper}


%% local definitions for section multiple merger coalescents
 \newcommand{\be}{\begin{equation}}
 \newcommand{\ee}{\end{equation}}
 \newcommand{\bd}{\begin{displaymath}}
 \newcommand{\ed}{\end{displaymath}}
\newcommand{\IN}{\ensuremath{\mathds{N}}}%
\newcommand{\EE}[1]{\ensuremath{\mathds{E}\left[ #1 \right]}}%
\newcommand{\one}[1]{\ensuremath{\mathds{1}_{\left\{ #1 \right\}}}}%
\newcommand{\prb}[1]{\ensuremath{\mathds{P}\left( #1 \right) } }%

\NewEnviron{esplit}[1]{%
\begin{equation}
\label{#1}
\begin{split}
  \BODY
\end{split}\end{equation}
}

\setcounter{secnumdepth}{2} % Print out appendix section numbers

\appendix

\end{document}
